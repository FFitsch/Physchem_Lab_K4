\documentclass[titlepage]{article}
\usepackage{fancyhdr}
\usepackage[margin=1.2in]{geometry}

\usepackage[utf8]{inputenc}
\usepackage[english]{babel}
\usepackage{csquotes}

\usepackage[backend=biber,style=verbose-ibid,sorting=ynt]{biblatex}
\addbibresource{W2.bib}

\usepackage[breaklinks]{hyperref}

\usepackage{graphicx}
\usepackage{float}
\usepackage[tableposition=top, justification=centering]{caption}

\usepackage{derivative}

\usepackage{array,tabularx,calc}
\newlength{\conditionwd}
\newenvironment{conditions}[1][where:]{
        #1\tabularx{\linewidth-\widthof{#1}}[t]
        {>{$}l<{$} @{${}={}$} X@{}}}
  {\endtabularx\\[\belowdisplayskip]}


% Title Page -----------------------------------------------------------
\title{Protocol \\ K4 - Oxalic Acid Oxidation}
\author{Group F\\Jonas Adamer (12225913)\\Florian Fitsch (12218283)\\Leonhard Ritt (12208881)}
\date{Date of experiment: 2024/10/21\\Date of submission:}

\pagestyle{fancy}
\fancyhf{}
\fancyhead[R]{\thepage}
\fancyfoot[L]{K4 Oxalic Acid Oxidation}
\fancyfoot[C]{Group F}
\fancyfoot[R]{Adamer, Fitsch, Ritt}

\begin{document}

% TITLE PAGE
\maketitle
\thispagestyle{empty}

% TABLE OF CONTENTS
\newpage
\tableofcontents
\thispagestyle{fancy}


\newpage
\section{Objective}
In this assignment, the kinetics of chemical reactions are studied through two separate experiments.

In the first experiment, the redox reaction between oxalic acid and potassium permanganate is studied: After determining the absorbance maximum of potassium permanganate, a series of solutions is produced and measured as a calibration for concentration of permanganate as a function of absorbance. Finally, the change in permanganate-concentration is measured in-situ, both with and without a catalyst, in order to determine the reaction order.

In the second experiment, the acid-base reaction between phenolphthalein and sodium hydroxide is studied. The absorbance of the reaction mixture is also measured in-situ and the reaction order regarding phenolphthalein determined.

\section{Experiment}
\subsection{Determination of Absorbance Maximum of KMnO\texorpdfstring{\textsubscript{4}}{4}} \label{ssec_experiment_abs_maximum}
In order to find the absorbance maximum, 79~mg of potassium permanganate were weighed out, dissolved in deionized wate and diluted in a 50~mL volumetric flask, yielding a 0.01~mol/L solution. 5~mL of this solution were taken out and once again diluted in a 50~mL flask, yielding a concentration of 1~mmol/L.

2~mL of the prepared solution were filled into a cuvette and inserted into the photometer. Its absorbance was then measured in wavelength increments of 10~nm between 450~nm and 600~nm. For each wavelength, the photometer was calibrated by setting values of 0\% and 100\% transmission, using an opaque block and a cuvette filled with deionized water, respectively. Since high absorbances were measured at 530~nm and 550~nm, another series of measurements was taken between 520~nm and 560~nm with increments of 2~nm. Using this process, a maximum at 532~nm was determined.

\subsection{Creation of Calibration Curve for KMnO\texorpdfstring{\textsubscript{4}}{4} Concentration}
To create the calibration, the 1~mmol/L potassium permanganate solution, which was prepared in Section \ref{ssec_experiment_abs_maximum}, was further diluted to prepare six further solutions, with concentrations of 0.75, 0.60, 0.50, 0.40, 0.25 and 0.10~mmol/L. After calibrating the photometer as described in Section \ref{ssec_experiment_abs_maximum}, the absorbance values of the seven solutions were measured at the previously determined wavelength of 532~nm. To account for inaccuracies, each measurement was taken three times, with the cuvette being filled with fresh solution after every measurement.

\subsection{Measurement of the Autocatalyzed Redox Reaction of KMnO\texorpdfstring{\textsubscript{4}}{4} and Oxalic Acid} \label{ssec_experiment_reaction_autocatalyzed}
By further diluting the 10~mmol/L solution prepared in \ref{ssec_experiment_abs_maximum}, a solution with a concentration of 4~mmol/L was prepared. Additionally, a 40~mmol/L solution of oxalic acid was obtained by dissolving AAAAA~g oxalic acid in deionized water in a AAAAA~mL volumetric flask. Additionally, a sulfuric acid solution of about 24~w\% was prepared by combining three parts water with one part concentrated (96~\%) sulfuric acid.

To start the reaction, 10~mL of the oxalic acid solution, 0.4~mL of the sulfuric acid and 4~mL of the potassium permanganate solution were added to a beaker and mixed thoroughly. A stopwatch was started once the permanganate solution was added. After filling 2~mL of the mixture into a cuvette, absorbance measurements were taken every 10 seconds, the measured value reached 0 or stopped changing. This experiment was repeated two more times.

\subsection{Measurement of the Mn\texorpdfstring{\textsuperscript{2+}}{2+}-Catalyzed Redox Reaction of KMnO\texorpdfstring{\textsubscript{4}}{4} and Oxalic Acid}
By dissolving AAAAA~g manganese sulfate in deionized water in a AAAAA~mL volumetric flask, a 40~mmol/L soluton was prepared.

Three reactions were measured as described in Section \ref{ssec_experiment_reaction_autocatalyzed}, with the addition of adding 0.2~mL of the prepared manganese sulfate soltion to the reaction mixture before measuring it.

\subsection{Measurement of the Deprotonation of Phenolphthalein by Sodium Hydroxide}
By diulting 7.5~mL of 2~mol/L sodium hydroxide with deionized water in a 50~mL volumetric flask, 50~mL of a 0.3~mol/L solution was prepared. Additionally, by dissolving 877~g of sodium chloride in deionized water and diluting to 50~mL, a 0.3~mol/L solution was prepared.

By diluting the former with the latter solution, small amounts of solutions with sodium hydroxide concentrations of 0.2, 0.1 and 0.05~mol/L, were prepared.

2~mL of each of the four solutions were then transferred into a cuvette, and a drop of phenolphthalein was added. The cuvettes were inserted into the photometer, which was set at 532~nm\footnote{Due to an error, the desired wavelength of 550~nm was not set before starting the experiment}, and, after dropping to an absorbance value of 0.8, further measurements were taken every 30 seconds over a span of six minutes.

\newpage
\section{Results and Discussion}
\subsection{Absorbance Maximum and Calibration Curve of KMnO\texorpdfstring{\textsubscript{4}}{4}}
The recorded absorbances of potassium permanganate are plotted as an absorbance spectrum in Figure \ref{fig_kmno4_absorbance_spectrum}. The determined absorbance maximum is at 532~nm.
%
\begin{figure}[H]
    \centering
    \includegraphics[width=0.7\textwidth]{Figures/Absorption\_Maximum.png}
    \caption{Absorbance spectrum of KMnO\textsuperscript{4} at wavelengths between 450 and 600~nm. The blue dots show the measurements of the first scan, which was taken in 10~nm increments, while the orange dots mark the measurements of the second scan (2~nm increments). The absorbance maximum at 532~nm is shown in green.}
    \label{fig_kmno4_absorbance_spectrum}
\end{figure}
%
\noindent Figure \ref{fig_kmno4_absorbance_calibration} shows the calibration curve (created through linear regression) for the concentration of potassium permanganate as a function of absorbance.

\begin{figure}[H]
    \centering
    \includegraphics[width=0.7\textwidth]{Figures/Calibration_Curve.png}
    \caption{Average values of measured absorbance plotted against potassium permanganate concentration. A calibration curve is fit to the data using linear regression.}
    \label{fig_kmno4_absorbance_calibration}
\end{figure}

By inserting the resulting linear equation into the Beer-Lambert law and rearranging, an equation for the molar attenuation coefficient \(\varepsilon\) can be formed:

\begin{equation}
  A = \varepsilon \cdot c \cdot d = 2.4919 \cdot c
\end{equation}
\begin{conditions}
    A & Absorbance \\
    \varepsilon & Molar attenuation coefficient \\
    c & Concentration of sample [mol/L] \\
    d & Path length through sample
\end{conditions}

\end{document}
