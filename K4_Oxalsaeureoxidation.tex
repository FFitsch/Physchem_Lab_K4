\documentclass[titlepage]{article}
\usepackage{fancyhdr}
\usepackage[margin=1.2in]{geometry}

\usepackage[utf8]{inputenc}
\usepackage[english]{babel}
\usepackage{csquotes}

\usepackage[backend=biber,style=verbose-ibid,sorting=ynt]{biblatex}
\addbibresource{K4.bib}

\usepackage[breaklinks]{hyperref}

\usepackage{graphicx}
\usepackage{float}
\usepackage[tableposition=top, justification=centering]{caption}
\renewcommand{\arraystretch}{2}

\usepackage{derivative}

\usepackage{array,tabularx,calc}
\newlength{\conditionwd}
\newenvironment{conditions}[1][where:]{
        #1\tabularx{\linewidth-\widthof{#1}}[t]
        {>{$}l<{$} @{${}={}$} X@{}}}
  {\endtabularx\\[\belowdisplayskip]}


% Title Page -----------------------------------------------------------
\title{Protocol \\ K4 - Oxalic Acid Oxidation}
\author{Group F\\Jonas Adamer (12225913)\\Florian Fitsch (12218283)\\Leonhard Ritt (12208881)}
\date{Date of experiment: 2024/10/21\\Date of submission:}

\pagestyle{fancy}
\fancyhf{}
\fancyhead[R]{\thepage}
\fancyfoot[L]{K4: Oxalic Acid Oxidation}
\fancyfoot[C]{Group F}
\fancyfoot[R]{Adamer, Fitsch, Ritt}

\begin{document}

% TITLE PAGE
\maketitle
\thispagestyle{empty}

% TABLE OF CONTENTS
\newpage
\tableofcontents
\thispagestyle{empty}

\addtocounter{page}{-1}

\newpage
\section{Objective}
In this assignment, the kinetics of chemical reactions are studied through two separate experiments.

In the first experiment, the redox reaction between oxalic acid and potassium permanganate is studied: After determining the absorbance maximum of potassium permanganate, a series of solutions is produced and measured as a calibration for concentration of permanganate as a function of absorbance. Finally, the change in permanganate-concentration is measured in-situ, both with and without a catalyst, in order to determine the reaction order.

In the second experiment, the acid-base reaction between phenolphthalein and sodium hydroxide is studied. The absorbance of the reaction mixture is also measured in-situ and the reaction order regarding phenolphthalein is determined.

\section{Experiment}
\subsection{Determination of Absorbance Maximum of KMnO\texorpdfstring{\textsubscript{4}}{4}} \label{ssec_experiment_abs_maximum}
In order to find the absorbance maximum, 79~mg of potassium permanganate were weighed out, dissolved in deionized water and diluted in a 50~mL volumetric flask, yielding a 0.01~mol/L solution. 5~mL of this solution were taken out and once again diluted in a 50~mL flask, yielding a concentration of 1~mmol/L.

2~mL of the prepared solution were filled into a cuvette and inserted into the photometer. Its absorbance was then measured in wavelength increments of 10~nm between 450~nm and 600~nm. For each wavelength, the photometer was calibrated by setting values of 0\% and 100\% transmission, using an opaque block and a cuvette filled with deionized water, respectively. Since high absorbances were measured at 530~nm and 550~nm, another series of measurements was taken between 520~nm and 560~nm with increments of 2~nm. Using this process, a maximum at 532~nm was determined.

\subsection{Creation of Calibration Curve for KMnO\texorpdfstring{\textsubscript{4}}{4}}
To create the calibration, the 1~mmol/L potassium permanganate solution, which was prepared in Section \ref{ssec_experiment_abs_maximum}, was further diluted to prepare six further solutions, with concentrations of 0.75, 0.60, 0.50, 0.40, 0.25 and 0.10~mmol/L. After calibrating the photometer as described in Section \ref{ssec_experiment_abs_maximum}, the absorbance values of the seven solutions were measured at the previously determined wavelength of 532~nm. To account for inaccuracies, each measurement was taken three times, with the cuvette being filled with fresh solution after every measurement.

\subsection{Measurement of the Autocatalyzed Redox Reaction of KMnO\texorpdfstring{\textsubscript{4}}{4} and Oxalic Acid} \label{ssec_experiment_reaction_autocatalyzed}
By further diluting the 10~mmol/L solution prepared in \ref{ssec_experiment_abs_maximum}, a solution with a concentration of 4~mmol/L was prepared. Additionally, a 40~mmol/L solution of oxalic acid was obtained by dissolving 0.505~g oxalic acid dihydrate in deionized water using a 100~mL volumetric flask. Additionally, a sulfuric acid solution of about 24~w\% was prepared by combining three parts water with one part concentrated (96~\%) sulfuric acid.

To start the reaction, 10~mL of the oxalic acid solution, 0.4~mL of the sulfuric acid and 4~mL of the potassium permanganate solution were added to a beaker and mixed thoroughly. A stopwatch was started once the permanganate solution was added. After filling 2~mL of the mixture into a cuvette, absorbance measurements were taken every 10 seconds, until the measured value reached 0 or stopped changing. This experiment was repeated two more times.

\subsection{Measurement of the Mn\texorpdfstring{\textsuperscript{2+}}{2+}-Catalyzed Redox Reaction of KMnO\texorpdfstring{\textsubscript{4}}{4} and Oxalic Acid}
By dissolving 0.340~g manganese sulfate in deionized water in a 50~mL volumetric flask, a 40~mmol/L soluton was prepared.

Three reactions were measured as described in Section \ref{ssec_experiment_reaction_autocatalyzed}, with the addition of adding 0.2~mL of the prepared manganese sulfate soltion to the reaction mixture before measuring it.

\subsection{Measurement of the Deprotonation of Phenolphthalein by Sodium Hydroxide}
By diulting 7.5~mL of 2~mol/L sodium hydroxide with deionized water in a 50~mL volumetric flask, 50~mL of a 0.3~mol/L solution was prepared. Additionally, by dissolving 877~g of sodium chloride in deionized water and diluting to 50~mL, a 0.3~mol/L solution was prepared.

By diluting the former with the latter solution, small amounts of solutions with sodium hydroxide concentrations of 0.2, 0.1 and 0.05~mol/L, were prepared.

2~mL of each of the four solutions were then transferred into a cuvette, and a drop of phenolphthalein was added. The cuvettes were inserted into the photometer, which was set at 532~nm\footnote{Due to an error, the desired wavelength of 550~nm was not set before starting the experiment}, and, after dropping to an absorbance value of 0.8, further measurements were taken every 30 seconds over a span of six minutes. Since the time available for the experiment ran out, only one sample of each concentration was measured. The final sample (0.05~mol/L) did not reach the starting absorbance of 0.8 in time, and as such, higher absorbance values were recorded and used for the evaluation of this sample.

\newpage
\section{Results and Discussion}
\subsection{Absorbance Maximum and Calibration Curve of KMnO\texorpdfstring{\textsubscript{4}}{4}}
The recorded absorbances of potassium permanganate are plotted as an absorbance spectrum in Figure \ref{fig_kmno4_absorbance_spectrum}. The determined absorbance maximum is at 532~nm.
%
\begin{figure}[H]
    \centering
    \includegraphics[width=0.68\textwidth]{Figures/Absorption\_Maximum.png}
    \caption{Absorbance spectrum of KMnO\textsubscript{4} at wavelengths between 450 and 600~nm. The blue dots show the measurements of the first scan, which was taken in 10~nm increments, while the orange dots mark the measurements of the second scan (2~nm increments). The absorbance maximum at 532~nm is shown in green.}
    \label{fig_kmno4_absorbance_spectrum}
\end{figure}
%
\noindent Figure \ref{fig_kmno4_absorbance_calibration} shows the calibration curve (created through linear regression) for the concentration of potassium permanganate as a function of absorbance. As can be seen, linearity, and thus the validity of the Lambert-Beer law is given for the entire tested concentration range.

\begin{figure}[H]
    \centering
    \includegraphics[width=0.68\textwidth]{Figures/Calibration_Curve.png}
    \caption{Average values of measured absorbance plotted against potassium permanganate concentration. A calibration curve is fit to the data using linear regression.}
    \label{fig_kmno4_absorbance_calibration}
\end{figure}
%
\noindent By inserting the resulting linear equation \(A = 2491.9 \cdot c\) into the Beer-Lambert law (Equation \ref{eq_beer_lambert}), rearranging and inserting the known cuvette length of 1~cm, the molar attenuation coefficient \(\varepsilon\) can be calculated for a wavelength of 532~nm:
%
\begin{equation} \label{eq_beer_lambert}
  A = \varepsilon \cdot c \cdot d = 2491.9 \cdot c
\end{equation}
\begin{equation} \label{eq_attenuation_coefficient}
  \varepsilon = \frac{2491.9}{d} = 2491.9
\end{equation}
\begin{conditions}
    A & Absorbance \\
    \varepsilon & Molar attenuation coefficient [L mol\textsuperscript{-1} cm\textsuperscript{-1}] \\
    c & Concentration of sample [mol/L] \\
    d & Path length through sample (cuvette length) [cm]
\end{conditions}

\subsection{Determination of Reaction Order of Redox Reaction}
By rearranging the Lambert-Beer law (Equation \ref{eq_beer_lambert}), the concentration of a sample can be determined from a given absorption value.
%
\begin{equation} \label{eq_concentration_from_attenuation}
    c = \frac{A}{\varepsilon \cdot d} = \frac{A}{2491.9}
\end{equation}
%
The resulting concentrations can then be plotted according to the rate laws of zeroth, first and second order, which are shown in Table \ref{tb_rate_laws}.
%
\begin{table}[H]
    \centering
    \caption{Rate equations and their integrated forms for reactions of zeroth, first and second order.}
    \label{tb_rate_laws}
    \begin{tabular}{|c|c|c|}
        \hline
        \textbf{0th Order} & \textbf{1st Order} & \textbf{2nd Order}
        \\
        \hline
        \(\displaystyle -\odv{c}{t} = k\) & \(\displaystyle -\odv{c}{t} = k \cdot c\) & \(\displaystyle -\odv{c}{t} = k \cdot c^2\)
        \\
        \hline
        \(\displaystyle c = c_0 - k \cdot t\) & \(\displaystyle \ln{c} = ln{c_0} - k \cdot t\) & \(\displaystyle \frac{1}{c} = \frac{1}{c_0} + k \cdot t\)
        \\
        \hline
    \end{tabular}
\end{table}
%
\noindent The following Figures \ref{fig_kmno_autocatalyzed_rates} and \ref{fig_kmno_mn_catalyzed_rates} show the respective concentrations during the autocatalyzed and Mn\textsuperscript{2+}-catalyzed redox reactions of oxalic acid with potassium permanganate plotted against time according to the integrated rate equations shown above.
%
\begin{figure}[H]
    \centering
    \begin{minipage}[c]{0.32\textwidth}
        \includegraphics[width=\textwidth]{Figures/autocatalyzed_0th_order.png}
    \end{minipage}
    \begin{minipage}[c]{0.32\textwidth}
        \includegraphics[width=\textwidth]{Figures/autocatalyzed_1st_order.png}
    \end{minipage}
    \begin{minipage}[c]{0.32\textwidth}
        \includegraphics[width=\textwidth]{Figures/autocatalyzed_2nd_order.png}
    \end{minipage}
    \caption{Measured concentrations during the first of three autocatalyzed redox reactions plotted against time according to the integrated rate equations of zeroth order (left), first order (center) and second order (right).}
    \label{fig_kmno_autocatalyzed_rates}
\end{figure}
%
\begin{figure}[H]
    \centering
    \begin{minipage}[c]{0.32\textwidth}
        \includegraphics[width=\textwidth]{Figures/mn_catalyzed_0th_order.png}
    \end{minipage}
    \begin{minipage}[c]{0.32\textwidth}
        \includegraphics[width=\textwidth]{Figures/mn_catalyzed_1st_order.png}
    \end{minipage}
    \begin{minipage}[c]{0.32\textwidth}
        \includegraphics[width=\textwidth]{Figures/mn_catalyzed_2nd_order.png}
    \end{minipage}
    \caption{Measured concentrations during the first of three Mn\textsuperscript{2+}-catalyzed redox reactions plotted against time according to the integrated rate equations of zeroth order (left), first order (center) and second order (right).}
    \label{fig_kmno_mn_catalyzed_rates}
\end{figure}
%
\noindent Comparing the two figures above - in particular the graphs of zeroth order - the differences between the studied types of catalysis become apparent: while the autocatalyzed reaction proceeds very slowly at first before speeding up as time proceeds and slowing down as the permanganate concentration approaches zero, the Mn\textsuperscript{2+}-catalyzed reaction shows a rather steady reaction speed, plummeting towards the end.

While such a steady decline in the zeroth-order-graph may be considered as an indicator of a zeroth-order reaction, the sudden drop in reaction speed of this line, contrasted with the rather steady rate shown in the first-order graph, might indicate a first-order reaction. Additionally, the fact that formation of Mn\textsuperscript{2+}-ions and thus autocatalysis of the reaction still takes place, might explain the lack of concavity of the zeroth-order-graph and the visible hump in the first-order-graph.

Additionally, from a chemical perspective, a second-order reaction might be expected due to the fact that two separate reactants are involved - however due to the high excess of oxalic acid (10 times the stoichiometric amount), the system might more accurately follow first-order-kinetics, resulting in a pseudo first-order reaction.

\subsection{Determination of Reaction Order of Acid-Base Reaction}
In order to determine the reaction order, the measured absorbance (which is proportional to the concentration of phenolphthalein) was plotted against time accoring to the integrated rate equations shown in Table \ref{tb_rate_laws}.

\begin{figure}[H]
    \centering
    \begin{minipage}[c]{0.32\textwidth}
        \includegraphics[width=\textwidth]{Figures/h2p_0th_order.png}
    \end{minipage}
    \begin{minipage}[c]{0.32\textwidth}
        \includegraphics[width=\textwidth]{Figures/h2p_1st_order.png}
    \end{minipage}
    \begin{minipage}[c]{0.32\textwidth}
        \includegraphics[width=\textwidth]{Figures/h2p_2nd_order.png}
    \end{minipage}
    \caption{Measured absorbance values during the reactions of phenolphthalein with different concentrations of sodium hydroxide plotted against time according to the integrated rate equations of zeroth order (left), first order (center) and second order (right).}
    \label{fig_h2p_rates}
\end{figure}
%
\noindent From the above figure one can see quite clearly that the first-order rate equation shows the most linear trend and, as such, the reaction is presumed to be a first-order reaction. This is due to the fact that the concentration of sodium hydroxide is several orders of magnitude greater than that of phenolphthalein and can thus be considered static over the course of the experiment. As such, the reaction is of pseudo-first-order. Nonetheless, it is apparent that, with higher sodium hydroxide-concentrations, the reaction proceeds at a higher rate, which further confirms that the reaction is not strictly first-order.

\end{document}
